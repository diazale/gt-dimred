\documentclass{article}
\setlength\parindent{0pt}

\begin{document}
Dear Editor,
\\\\
We are pleased to resubmit our manuscript \emph{UMAP reveals cryptic population structure and phenotype heterogeneity in large genomic cohorts} for publication in PLOS Genetics. Since our previous submission we have thoroughly edited the manuscript to address reviewer comments in detail, to provide guidance on usage and interpretation of UMAP, and to provide rigorous follow up on findings.
\\\\
Among the findings are differences between East Asian populations in the distributions of FEV1, a clinically important measurement used in studies of lung function, which have not been identified to date to the best of our knowledge. We have also identified more detailed population structure patterns using the country of birth for individuals in the UK biobank and demonstrated how to use UMAP to identify patterns in ancestry on a global scale.
\\\\
The methods described in our preprint continue to be used to study ancestral diversity, such as in the National Geographic Genographic Project\footnote{Dai, Chengzhen L., et al. "Population histories of the United States revealed through fine-scale migration and haplotype analysis." BioRxiv (2019): 577411.} and by 23andMe\footnote{23andMe. https://blog.23andme.com/ancestry/23andme-tests-new-ancestry-breakdown-in-central-and-south-asia/. ``23andMe Tests New Ancestry Breakdown in Central and South Asia'' (2019-04-03)}, whose interpretation of the population structure in the Gujarati is now part of the current manuscript. 
\\
\\
Sincerely,
\\
Alex Diaz-Papkovich and Simon Gravel
\\
on behalf of all authors.

\end{document}