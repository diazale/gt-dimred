\documentclass{article}
\setlength\parindent{0pt}

\begin{document}
Dear Editor,
\\\\
We are pleased to submit our manuscript \emph{UMAP reveals cryptic population structure and phenotype heterogeneity in large genomic cohorts} for publication in PLOS Genetics. Using a newly developed dimensional reduction method called UMAP, this work provides an efficient strategy to identify and visualize population structure in massive and heterogeneous genomic datasets. In contrast to existing methods, this approach is able to represent both discrete and continuous population structure, as well as admixture. 
\\\\
By applying UMAP to large contemporary human genetic datasets (the 1000 Genomes Project, the Health and Retirement Study, and the UK biobank), we reveal a wide range of patterns that had never been observed in these well-studied cohorts. For example, we identify a cluster of Hispanic individuals that is largely restricted to the US Mountain region. These individuals are likely descendants of the Hispano population. We also identify multiple associations between genetics, geography, and phenotypes in the UK biobank and population structure within 1000 Genomes Project populations.
\\\\  
In other -omics fields, such as single-cell expression data, nonlinear nonparametric approaches such as t-SNE have become standard tools for data exploration. Principal component analysis has remained the standard for dimension reduction in human genetics, partly because nonlinear alternatives such as t-SNE don't behave well when applied to large-scale genetic data. We show that UMAP outperforms t-SNE computationally, qualitatively and quantitatively, and to such an extent that we expect it to become a standard companion to principal component analysis. 
\\\\
In the few weeks since our preprint was published, leading cohorts such as GnomAD have started using UMAP to represent their data\footnote{Fancioli, Laurent. ``gnomAD v2.1''. https://macarthurlab.org/2018/10/17/gnomad-v2-1/ (2018-10-17)}, and it has also been integrated as a preprocessing step in Machine Learning approaches\footnote{Tonkin-Hill, Gerry, et al. ``Fast Hierarchical Bayesian Analysis of Population Structure.'' bioRxiv (2018): 454355.}. Our preprint is also in the 99th percentile of all-time bioRxiv preprints for the Altmetric impact statistic. 
\\\\
We believe that this will be useful and of interest to a broad variety of geneticists including population genetics, medical genetics, anthropology, and data science, which makes the manuscript an ideal candidate for PLOS Genetics.
\\
\\
Sincerely,
\\
Alex Diaz-Papkovich and Simon Gravel
\\
on behalf of all authors.

\end{document}