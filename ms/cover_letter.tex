\documentclass{article}
\setlength\parindent{0pt}

\begin{document}
Dear Editor,
\\\\
We are pleased to submit our manuscript \emph{UMAP reveals cryptic population structure and phenotype heterogeneity in large genomic cohorts} for publication in Nature Genetics. Using a recently developed dimensionality reduction method called UMAP, this work provides an efficient strategy to identify and visualize population structure in the massive and heterogeneous genomic datasets that are becoming the new frontier of genomics research. In contrast to existing methods, this approach is able to represent both discrete and continuous population structure and the distributions of ancestries, admixture, and phenotypes within the context of such structure. \\\\
By applying UMAP to large contemporary human genetic datasets (the 1000 Genomes Project, the Health and Retirement Study, and the UK biobank), we reveal a wide range of patterns that had never been observed in these well-studied cohorts. We identify a cluster of Hispanic individuals that is largely restricted to the US Mountain region --- likely descendants of the Hispano population --- and  multiple associations between genetics, geography, and phenotypes in the UK biobank as well as population structure within 1000 Genomes Project populations.
%\\\\
%In other -omics fields, such as single-cell expression data, nonlinear nonparametric approaches such as t-SNE have become standard tools for data exploration. Principal component analysis has remained the standard for dimension reduction in human genetics, partly because nonlinear alternatives such as t-SNE don't behave well when applied to large-scale genetic data. We show that UMAP outperforms t-SNE computationally, qualitatively and quantitatively, and to such an extent that we expect it to become a standard companion to principal component analysis. 
\\\\
We show that UMAP outperforms t-SNE --- the current state-of-the-art --- computationally, qualitatively, and quantitatively, and to such an extent that we expect it to become a standard companion to principal component analysis.
Methods described in our preprint have already been used to study ancestral diversity in the Genome Aggregation Database (gnomAD)\footnote{Karczewski, Konrad J., et al. "Variation across 141,456 human exomes and genomes reveals the spectrum of loss-of-function intolerance across human protein-coding genes." BioRxiv (2019): 531210.} and the National Geographic Genographic Project\footnote{Dai, Chengzhen L., et al. "Population histories of the United States revealed through fine-scale migration and haplotype analysis." BioRxiv (2019): 577411.}, and by 23andme. Indeed, 23andme was able to follow up on one of the population structure signals we identified in our preprint by recontacting their participants, and the resulting interpretation of the population structure in the Gujarati is now part of the current manuscript. 

We believe that this is compelling evidence that UMAP will be useful and of interest to a broad variety of geneticists in population genetics, medical genetics, anthropology, and data science, which makes the manuscript an ideal candidate for Nature Genetics.
\\
\\
Sincerely,
\\
Alex Diaz-Papkovich and Simon Gravel
\\
on behalf of all authors.

\end{document}